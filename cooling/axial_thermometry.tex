% -*- mode: latex; fill-column: 65; -*-
\documentclass[aps, pra, preprint]{revtex4-1}
\usepackage{amssymb}
\usepackage{amsmath}
\usepackage{siunitx}
\usepackage[colorinlistoftodos,prependcaption,textsize=tiny]{todonotes}
\usepackage{blindtext}
\presetkeys%
    {todonotes}%
    {inline,backgroundcolor=yellow}{}

\frenchspacing

\bibliographystyle{apsrev4-1}

\begin{document}

\title{Determination of temperatures of the axial modes of an ion
  crystal in a Penning trap}
\author{Dominic Meiser}
\affiliation{Trimble Inc, Boulder, 4730 Walnut Street, Suite 201,
  Boulder, CO 80301, USA.}

\begin{abstract}
  In these notes we discuss how to extract the temperatures of
  the drum-head modes of a crystal of ultra-cold ions in a
  Penning trap from spectra of the axial motion.
\end{abstract}

\maketitle


\section{The drum head modes of an ion crystal}

We consider an ensemble of $N$ ions in a Penning trap with the
magnetic field aligned with the $z$ axis. We assume that the
motion of the ions along the $z$ direction is decoupled from the
other degrees of freedom. We consider the $z$ degrees of
freedom only.

The Hamiltonian is given by
\begin{equation}
  H=
    \sum_{j=1}^N\frac{p_j^2}{2m_j} +
    \sum_{j=1}^NV_1(z_j) +
    \sum_{\substack{i=1\\j = i + 1}}^{N}V_2(z_i - z_j)\;.
\end{equation}
In this equation, $p_j$ is the $z$-component of the momentum of
ion $j$. The single particle potential $V_1$ represents the
electrostatic trap potentials, and the pairwise interactions
$V_2$ are due to the Coulomb interaction. 

Expanding the Hamiltonian to second order around the stable
steady state at $z_j=0$, we obtain
\begin{equation}
  H\approx 
    \sum_{j=1}^N \frac{p_j^2}{2m_j} +
    \frac{1}{2}\sum_{j=1}^N m_j \omega_z^2 z_j^2+
    \frac{1}{2}\sum_{\substack{i = 1\\j = i + 1}}^{N} V_{ij}(z_i-z_j)^2\;.
\end{equation}

Introducing the vectors
\begin{equation}
  z=\left[
    \begin{array}{c}
      z_1\\
      z_2\\
      \vdots
    \end{array}
  \right]
\end{equation}
and
\begin{equation}
  p=\left[
    \begin{array}{c}
      p_1\\
      p_2\\
      \vdots
    \end{array}
  \right]
\end{equation}
along with the matrices
\begin{equation}
  M=\left[
    \begin{array}{ccc}
      m_1&0&\hdots\\
      0&m_2&\\
      \vdots&&\ddots
    \end{array}
  \right]
\end{equation}
and
\begin{equation}
  V=\left[
    \begin{array}{cccc}
      m_1\omega_z^2&V_{1,2}&V_{1,3}&\hdots\\
      V_{2,1}&m_2\omega_z^2&V_{2,3}&\hdots\\
      V_{3,1}&V_{3,2}&m_3\omega_z^2&\\
      \vdots&\vdots&&\ddots
    \end{array}
  \right]\;,
\end{equation}
we can write the linearized Hamiltonian as
\begin{equation}
  H =
  \frac{1}{2}\left\langle p, M^{-1}p \right\rangle +
  \frac{1}{2}\left\langle z, V z \right \rangle\;,
  \label{eqn:linearized_hamiltonian}
\end{equation}
with $\langle \cdot, \cdot\rangle$ the inner product.

The equations of motion are
\begin{eqnarray}
  \dot{z} &=& \frac{\partial H}{\partial p} = M^{-1}p\label{eqn:zdot}\\
  \dot{p} &=& -\frac{\partial H}{\partial z} = -Vz\label{eqn:pdot}\;.
\end{eqnarray}
Taking another time derivative of Eqn.~\eqref{eqn:zdot} and
eliminating $\dot{p}$ using Eqn.~\eqref{eqn:pdot} we find
\begin{equation}
  M\ddot{z}=-Vz\;.
\end{equation}
Making the ansatz
\begin{equation}
  z(t)=e^{i\omega t}z(0)
\end{equation}
we obtain the eigenvalue equation
\begin{equation}
  M\omega^2z = Vz\;.
\end{equation}
Since $M$ and $V$ are symmetrical we have a complete orthogonal
set of eigenvectors $\phi_1, \phi_2,\ldots$ with eignvalues
$\omega_1, \omega_2, \ldots$. We assume that the eigenmodes are
normalized according to
\begin{equation}
  \left\langle \phi_m,\phi_n\right \rangle = a^2\delta_{m,n}\;,
\end{equation}
where $a$ is the unit of length and $\delta_{m,n}$ is Kronecker's
delta symbol.


\section{Energy in terms of eigenmodes}

We can insert the expansion of the ion trajectories in terms of the
eigenmodes,
\begin{equation}
  z=\sum_{m=1}^N e^{i\omega_m t} \alpha_m \phi_m\;,
\end{equation}
into the Hamiltonian Eqn.~\eqref{eqn:linearized_hamiltonian}. A
simple calculation yields
\begin{equation}
  H=\sum_{m,n=1}^N\left| \alpha_m\right|^2
    \omega_m^2\left\langle \phi_m,M\phi_n \right\rangle\;.
\end{equation}
This can be further simplified in the case where all ions
have the same mass, $m=m_1=m_2=\ldots$. We now specialize our
discussion to that case. We obtain
\begin{equation}
  H=ma^2\sum_{m=1}^N\omega_m^2\left| \alpha_m \right|^2\;.
\end{equation}
Each mode contributes
\begin{equation}
  E_m=ma^2\omega_m^2\left| \alpha_m \right|^2\;,
  \label{eqn:ModeEnergy}
\end{equation}
to the total energy. Note that if we were to choose the harmonic
oscillator length, $a_{\rm osc}=\sqrt{\hbar/(m\omega_m)}$ as the
unit of length this reduces to the well known formula
\begin{equation}
  E_m=\hbar\omega_m\left| \alpha_m \right|^2\;.
\end{equation}
But the choice of unit of length is not essential to our discussion
here.


\section{Determination of mode energies from spectra}

From the dynamical simulations we compute the power spectral
density
\begin{equation}
  I(\omega) =
    \sum_{j=1}^N\left| \tilde{z}_j(\omega) \right|^2 +
    \sum_{j=1}^N\left| \tilde{z}_j(-\omega) \right|^2 =
    \left\langle \tilde{z}(\omega), \tilde{z}(\omega)\right \rangle +
    \left\langle \tilde{z}(-\omega), \tilde{z}(-\omega)\right \rangle 
    \;.\label{eqn:psd}
\end{equation}
In this equation, $\tilde{z}_j$ is the Fourier transform of the
trajectory of ion $j$,
\begin{equation}
  \tilde{z}_j(\omega) = \frac{1}{T}\int_0^T e^{-i\omega t}z_j(t)\, dt,
\end{equation}
with $T$ the total integration time of the simulation. Inserting
our mode expansion, the power spectral density becomes
\begin{equation}
  I(\omega) =
  \sum_{m=1}^N\frac{a^2 \left| \alpha_m \right|^2}{T^2}
  \int_0^T\int_0^T\left( e^{-i(\omega-\omega_m)(t-t^\prime)} +
    e^{i(\omega+\omega_m)(t-t^\prime} \right)\,dt\,dt^\prime\;.
\end{equation}
We now consider a single resonance $\omega_m$ and assume that
its resonance is well resolved. In that case we can integrate the
power spectral density over the resonance. We find
\begin{equation}
  \int_{\omega_m-\Delta}^{\omega_m+\Delta} I(\omega)\,d\omega =
  \frac{\pi a^2}{T}\left| \alpha_m \right|^2\;,
\end{equation}
where we have assumed that $\Delta\gg T^{-1}$. Solving for
$|\alpha_m|^2$ and inserting into the expression for the mode
energy, Eqn.~(\ref{eqn:ModeEnergy}), we find
\begin{equation}
  E_m = m\omega_m^2\frac{T}{\pi}
        \int_{\omega_m-\Delta}^{\omega_m+\Delta}I(\omega)\,d\omega\;.
  \label{eqn:ModeEnergyFromSpectrum}
\end{equation} 


\section{Note on discrete signals and discrete Fourier transform}

In the simulations we sample the ion motion with a discrete,
constant sampling rate $\Delta t$. In terms of this discretely
sampled signal, the Fourier transform can be approximatet by a
discrete Fourier transform (which we evaluate with the aid of a
Fast Fourier Transform). We can consider the discrete Fourier
transform as a Riemann sum approximation to the Fourier
transform,
\begin{equation}
  \tilde{z}(\omega)=
  T^{-1}\int_0^Te^{-i\omega t}z(t)\,dt \approx
  T^{-1}\sum_{n=0}^{N_s - 1}\Delta t e^{-i\omega n\Delta t}z(n \Delta t) =
  \frac{1}{N_s}\sum_{n=0}^{N_s-1}e^{-i\omega n\Delta t}z(n \Delta t)\;.
\end{equation}
In this equation, $N_s = T / \Delta t$ is the total number of
samples. Introducing the frequency grid spacing
$\Delta\omega=2\pi/T$ we can write the approximate Fourier
transform as
\begin{equation}
  \tilde{z}(l\Delta\omega)\approx
  \frac{1}{N_s}\sum_n^{N_s}e^{-i2\pi ln/N_s}z(n \Delta t)\;.
\end{equation}
We have to keep in mind that frequencies exceeding the Nyquist
frequency, $l\Delta\omega>\pi/\Delta t$, wrap around to negative
frequencies. This appxoximation for $\tilde{z}(\omega)$ can now
be used in Eqn.~(\ref{eqn:ModeEnergyFromSpectrum}). To find the
mode energy we have to numerically evaluate the integral.
Currently we do this using the trapezoidal rule. However, several
refinements could be made and may be needed because the
resonances are sharp in the absence of laser cooling. For
example, we could imagine fitting a Lorentzian to the line
profile and then analytically evaluating the integral in
Eqn.~(\ref{eqn:ModeEnergyFromSpectrum}).



\end{document}
