\documentclass[aps, pra, preprint]{revtex4-1}
\usepackage[T1]{fontenc}
\usepackage{lmodern}
\usepackage{amssymb}
\usepackage{amsmath}
\usepackage{siunitx}
\usepackage[colorinlistoftodos,prependcaption,textsize=tiny]{todonotes}
\usepackage{blindtext}
\presetkeys%
    {todonotes}%
    {inline,backgroundcolor=yellow}{}
\usepackage{changepage}
\usepackage{setspace}
\bibliographystyle{apsrev4-1}


\newenvironment{comment}[1]{
    \begin{adjustwidth}{5cm}{-1.5cm}
        \singlespace
        \small
        [#1]
        \it
}
{
    \end{adjustwidth}
}



\begin{document}

\title{First principles simulations of the dynamics of ultra-cold
ions in Penning traps}

\author{Chen Tang}
\author{Scott E Parker}
\affiliation{Center for Integrated Plasma Studies,
    Department of Physics,
    University of Colorado at Boulder, CO 80309, USA.}
\author{Dominic Meiser}
\affiliation{Trimble Inc, Boulder, 4730 Walnut Street, Suite 201,
    Boulder, CO 80301, USA.}
\author{John J Bollinger}
\affiliation{National Institute of Standards and Technology,
    Boulder, Colorado 80305, USA.}


\begin{abstract}
    Abstract.
\end{abstract}

\maketitle


\section{Introduction}


\begin{comment}{DM}
    Point of intro: Establish ultra-cold ions in Penning traps as
    an interesting field of research. Show that modelling can
    make useful contributions in this area. At the intersection
    of plasma physics and quantum science. Answer the question
    "Why another code?"
\end{comment}


\begin{itemize}
    \item Ultra-cold ions in Penning traps enable interesting
        research in a number of different areas including
        quantum optics, quantum metrology, quantum
        simulation, quantum computing, and plasma physics.
    \item Cold temperatures are essential for many of these
        experiments.
    \item Modelling can help getting to lower temperatures.
        \begin{itemize}
            \item Understand cooling limits
            \item Optimize experimental parameters
        \end{itemize}
    \item But besides its utility for experiment, modelling is
        interesting in its own right. Ultra-cold ions are in an
        exotic regime characterized by very strong coupling
        ($\Gamma \gtrsim 100$). Benchmark for plasma simulation
        code that can then be used for the study of other
        strongly coupled plasmas.
    \item Review of modelling and simulation literature,
        state-of-the-art. Freericks, Torrisi, Dan Dubin, but also
        many others, especially in the molecular dynamics and
        plasma physics simulation area. Scott and John to add.
    \item In this paper we \ldots
    \item How do we go beyond previous appraoches? What
        approximations do we make? What assumptions do we avoid?
    \item Contributions:
        \begin{itemize}
            \item Verify and validate a unified, microscopically
                motivated code for the simulation of ultra-cold
                ions in Penning traps.
            \item Finite temperature theory of out-of-plane
                modes. Determination of mode temperatures.
                Investigation of thermal equilibrium of these
                modes.
            \item Dynamical study of the cooling of the in-plane
                degrees of freedom.
        \end{itemize}
\end{itemize}


\section{Computational approach: first principles molecular
dynamics simulations}

\begin{comment}{DM}
    A lot of this can be taken from the {\tt cooling.tex}
    notes. With some editing and polishing.
\end{comment}


\begin{itemize}
    \item Particle method
    \item In the lab frame
\end{itemize}


\subsection{Equations of motion}

\begin{itemize}
    \item Cyclotron motion and treatment of the magnetic
        field.
    \item Electro-static trap forces including rotating wall
        potential.
    \item Coulomb potential
\end{itemize}


\subsection{Operator splitting}

\begin{itemize}
    \item Discuss splitting
    \item Figure: Show convergence plot
    \item Discuss time scales
    \item Discuss precision
\end{itemize}


\subsection{Doppler cooling}

\begin{itemize}
    \item Describe physics of resonance fluorescence and
        Doppler cooling
    \item Figure: Show free space Doppler cooling
    \item Compare with theoretical result
\end{itemize}


\section{Results}


\subsection{Single plane instability}

This is a verification and validation result. The single plane to
two plane instability is well characterized experimentally.


\subsection{Axial spectra}

\begin{itemize}
    \item Figure: Show axial spectra plot with resonances from
        Freericks code overlayed.
    \item Compare with Freericks results
    \item Discuss mode temperatures
\end{itemize}


\subsection{Cooling dynamics}

\begin{itemize}
    \item Figure: Show temperatures as a function of time for a
        number of initial conditions. In-plane and
        out-of-plane temperatures.
    \item Figure: Compare with Torrisi (scan of offset, beam width, and
        laser detuning)
    \item
        Discuss where the two approaches agree and where they
        disagree. Explain disagreement as best we can.
\end{itemize}


\section{Conclusion}

\begin{itemize}
    \item In this paper we have \ldots
    \item In the future we will \ldots
\end{itemize}


\begin{acknowledgments}
    Funding agencies etc.

    DM thanks AFOSR and Tech-X Corp. for supporting the
    development of a similar molecular dynamics simulation code
    that inspired some of the design of the code presented here.
\end{acknowledgments}


\bibliography{md_cold_ions}


\end{document}
